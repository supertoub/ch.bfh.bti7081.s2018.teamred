\chapter{Introduction}

Das System wird benötigt im Umgang mit Angststörungen bzw. Sozialen \gls{Phobien}. Aktuell existieren keine Lösungen für Patienten in ambulanter Behandlung die an Angststörungen leiden. Unsere Applikation wird den Patienten bei der Genesung unterstützen und im mehr Lebensqualität geben. Auch soll es dem Patienten ermöglichen durch weniger Angst wieder ein produktives Mitglied der Gesellschaft zu werden.

\section{Kurzbeschreibung der Funktionen des Systems}

Unsere Applikation bietet über \gls{Gameification} einen Anreiz sich jeden Tag schrittweise an ein Ziel anzunähern. Die Applikation bietet die Möglichkeit, das der Patient zusammen mit dem Therapeuten \gls{Challenges} erstellt. Diese \gls{Challenges} sind Aufgaben welche der Patient anschliessend gestellt bekommt. Nach der Ausführung einer Challenge wird der Patient nach einer Beschreibung des Erlebnisses gefragt. Diese Bechreibungen sind für den Therapeuten zugänglich, später während den Therapiesitzungen können sie besprochen werden.

\section{Wie wird es mit anderen Systemen zusammen arbeiten}

Die Applikation wird selbst laufen bzw. keinen Zugang zu anderen Systemen benötigen. Es müssen keine Schnittstellen zu anderen Systemen für den Datenabgleich vorgesehen werden. \\ \\
Die Applikation komuniziert nicht mit anderen Systemen. Die Verbindungen welche projektiert werden, sind Emails oder Push Nachrichten an den Therapeuten, welche von unserer Applikation ausgelöst werden.
